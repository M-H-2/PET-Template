\documentclass[a4paper, 12pt]{report}
\usepackage{pentest-report}

\title{Penetrationstestbericht - \ac{DVWA}}
\author{Manuel Heckmann}
\date{\today}

\begin{document}
\maketitle
\tableofcontents
\listoffigures

\chapter*{Abkürzungsverzeichnis}
\begin{acronym}
    \acro{DVWA}{Damn Vulnerable Web Application}
    \acro{LPE}{Local Privilege Escalation}
    \acro{PET}{Penetration Testing}
    \acro{VM}{Virtuelle Maschine}
\end{acronym}

\chapter{Einführung}
Im Zuge der Vorlesung \ac{PET} an der Hochschule Mannheim im Sommersemester 2021
wurde den Studierenden die Aufgabe gestellt, einen Penetrationstestbericht zu verfassen.
Dies dient der Vorbereitung auf die Abschlussprüfung dieses Faches,
welche das Aufspüren und anschließende Dokumentieren von Schwachstellen einer
dem Studenten zugewiesenen virtuellen Maschine (\acs{VM}) umfasst.
Die Dokumentation wird in Form eines Penetrationstestberichts abgegeben werden.
Dieser Bericht soll der Vorbereitung auf diese Prüfung dienen.
Bei der konkret untersuchten Anwendung dieses Berichts handelt es sich um die \ac{DVWA}.

\ldots

\chapter{Technische Umgebung und Randbedingungen}
Bei dem Zielsystem handelt es sich um eine \ac{VM} auf der \emph{Metasploitable 2} installiert wurde.
\emph{Metasploitable 2} ist eine sogenannte \emph{VulnBox}; ein vorgefertigtes System, welches der Übung
von sicherheitsrelevanten Fertigkeiten, im Speziellen Penetrationstests, dient.
Wie bereits in der Einführung erwähnt, handelt es sich bei der untersuchten Anwendung um die \ac{DVWA},
welche ebenfalls diesem Trainingstyp zuzuschreiben ist auf dem zuvorgenannten \ac{VM}-System ausgeführt wird.
Wie der Name bereits verrät, handelt es sich bei der \ac{DVWA} um eine Web-Anwendung.
Die Angriffe werden dementsprechend über das Netzwerk ausgeführt.
Hierbei hat das Zielsystem eine von der Virtualisierungsumgebung VirtualBox zugewiesene dynamische IP-Adresse
im Subnetz \textbf{192.168.3.0/24}.
Im Zuge dieser Untersuchung handelt es sich bei der zugewiesenen Adresse um \textbf{192.168.3.4}.
Die untersuchte Webanwendung findet sich dementsprechend unter \url{http://192.168.3.4/dvwa/}.

\ldots

\chapter{Kategorisierung der Schwachstellen}

Im Zuge dieses Berichts werden Schwachstellen auf zweierlei Art dokumentiert:
\begin{enumerate}
    \item Tabellarisch
    \item In Prosaform
\end{enumerate}

Ersteres bildet eine Einführung und einen schnellen Überblick zu der jeweiligen Schwachstelle.
Eine Erklärung der Tabellenspalten findet sich auf der nächsten Seite.

Weiterhin folgen auf die Übersichtstabelle weitere Erläuterungen in Prosaform, sowie Bilder, sofern vorhanden.

\begin{center}
    \begin{tabularx}{\textwidth}{ | X | X | }
        \hline
        \rowcolor{lightgray} \textbf{Name} & \textbf{Erklärung} \\
        \hline
        Kategorie & Angabe der Mitre-CWE-Kategorisierung \footnote{nach \url{https://cwe.mitre.org}} \\
        \hline
        Einstufung & Schweregrad der Schwachstelle.
        
        \emph{Niedrig} bedeutet, dass die Schwachstelle für sich genommen kein Sicherheitsproblem darstellt,
        und nur in Ausnahmefällen die Systemsicherheit gefährden kann.
        
        \emph{Mittel} bedeutet, dass die Systemsicherheit in Kombinationen mit weiteren Schwachstellen schnell
        gefährdet sein kann; es kann sich also schnell zu einem schwerwiegenden Sicherheitsproblem entwickeln
        
        \emph{Schwer} bedeutet, dass es sich um einen schwerwiegenden Mangel handelt, der umgehend behoben werden sollte. \\
        \hline
        Auswirkung & Konkrete Gefährdung die sich aus der Schwachstelle ergibt \\
        \hline
        Empfohlene Maßnahme & Gibt eine Möglichkeit an, die Schwachstelle zu beheben \\
        \hline
        Notwendiges Angreifer-Niveau & Wie erfahren ein Angreifer sein muss, um die Schwachstelle ausnutzen zu können.
        
        \emph{Anfänger} bedeutet, dass ein automatisiertes Tool zu benutzen und/oder rudimentäre Kenntnisse ausreichend sind.

        \emph{Professional} bedeutet, dass ein Angreifer fundiertes Wissen und die entsprechenden Werkzeuge besitzen muss.
        
        \emph{Experte} bedeutet, dass ein Angreifer in der Lage sein muss, selbst Exploits entwickeln zu können und durch Quellcode-Analyse neue Schwachstellen zu finden \\
        \hline
    \end{tabularx}
\end{center}

\chapter{Verlauf und Ergebnis}
Im Wesentlichen wurden alle Exploits in der aufgeführten Reihenfolge durchgeführt.
Werkzeuge wurden nur verwendet, wo dies auch angegeben wurde.
An dieser Stelle ist besonders zu erwähnen, dass man bemerkenswert einfach über die
in Kapitel 5.6 beschriebene Schwachstelle Zugriff auf eine Shell des Zielsystems erhalten kann.
Um die nötigen Dateien für weitere Angriffe hochzuladen, stehen die erlangte Shell
und die in Kapitel 5.8 beschriebene Schwachstelle zur Verfügung.

\ldots

Sobald man Zugriff auf das System sowie die nötigen Exploits hochgeladen hat, muss man diese nur noch vorbereiten und ausführen.
Im Falle des genutzten \ac{LPE}-Exploits reicht ein Kompilieren mit \emph{gcc} sowie das anschließende Ausführen.

Das Ergebnis der Untersuchung ist, dass das System viele schwere Schwachstellen aufweist und schnellstmöglich abgesichert werden sollte.
Genauere Informationen zu den vorliegenden Problemen finden Sie in den Detailberichten im folgenden Kapitel.


\chapter{Identifizierte Schwachstellen}
\section{Benutzerdaten auf Anmeldeseite einsehbar}
\vulntable{\href{https://cwe.mitre.org/data/definitions/200.html}{CWE-200: Exposure of Sensitive Information to an Unauthorized Actor}}{Schwer}{Administrativer Zugriff auf das gesamte System}{Anmeldedaten nicht auf der Startseite einblenden, Anmeldedaten ändern}{Anfänger}

\begin{figure}
    \includegraphics[width=\textwidth, keepaspectratio]{admin_credentials.png}
    \caption{Die Administrator-Zugriffsdaten werden auf der Startseite jedem Besucher angezeigt}
\end{figure}

Betroffene URL: \url{http://192.168.3.4/dvwa/login.php} \\
Die Anmeldedaten für administrativen Zugriff sind auf der Startseite für jeden unverschlüsselt einsehbar.
Diese Zugangsdaten für den administrativen Zugriff sollten umgehend von der Startseite entfernt und geändert werden, da sonst jeder Besucher der Seite trivial
Zugriff auf die Administration der Webanwendung erlangt.

\section{Reflected Cross-Site-Scripting}
\label{rxss}
\vulntable{\href{https://cwe.mitre.org/data/definitions/79.html}{CWE-79: Improper Neutralization of Input During Web Page Generation ('Cross-site Scripting')}}{Mittel}{Ausführung von clientseitigem Code}{Eingaben säubern, z.B. durch Filtern von HTML-Sonderzeichen}{Anfänger}

Betroffene URL: \url{http://192.168.3.4/dvwa/vulnerabilities/xss_r/} \\

Der URL-Parameter \emph{name} ist angreifbar.
Die Eingabe wird direkt in das HTML der Seite eingebettet.
Somit kann ein Angreifer beliebigen Code clientseitig ausführen.
Ein mögliches Angriffszenario wäre das Stehlen des Session-Cookies.
Hiermit könnte ein Angreifer die Identität eines anderen Nutzers übernehmen.

Der folgende HTML-Quelltext genügt für das eben genannte Beispiel:
\lstinputlisting[language=html]{../res/src/xss.html}

\begin{figure}
    \includegraphics[width=\textwidth, keepaspectratio]{r_xss.png}
    \caption{Das Ergebnis der Eingabe des HTML-Quelltextes in das Eingabefeld}
\end{figure}

\begin{figure}
    \includegraphics[width=\textwidth, keepaspectratio]{xss_cookie.png}
    \caption{Das Ergebnis des Quelltextes aus Sicht eines HTTP-Servers des Angreifers. Man beachte die Cookies des Nutzers nach dem \emph{?}}
\end{figure}

Man kann nun den \emph{PHPSESSID}-Cookie benutzen, um die Sitzung des angegriffenen Benutzers zu übernehmen.

\ldots

\end{document}
